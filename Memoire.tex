\documentclass[12pt,a4paper,titlepage]{article}
\usepackage{longtable,geometry}
\geometry{dvips,a4paper,margin=1.5in}

\usepackage[french]{babel}
\usepackage[utf8]{inputenc}
\usepackage[T1]{fontenc}

\usepackage{eurosym}

\usepackage[babel]{csquotes}
\MakeAutoQuote{«}{»}

\usepackage{graphicx}
\graphicspath{ {./images/} }

\usepackage{wrapfig}
\usepackage{caption}

\DeclareCaptionLabelFormat{simpleNumber}{#2}

\usepackage[final]{pdfpages}

\usepackage{rotating}
\usepackage{rotfloat}
\usepackage{float}

\usepackage{longtable, tabu}

\usepackage{makeidx}

\usepackage{blindtext}
\usepackage{scrextend}
\usepackage[pdfpagelabels, plainpages=false]{hyperref}
\usepackage[figure]{hypcap}

\setcounter{tocdepth}{3}

\usepackage[backend=biber]{biblatex}
\addbibresource{bibliography.bib}


\usepackage[toc,acronym,xindy]{glossaries}
\makeglossaries
\makeindex

\usepackage{xparse}

\DeclareDocumentCommand{\newdualentry}{ O{} O{} m m m m } {
    \newglossaryentry{gls-#3}{name={#5},text={#5\glsadd{#3}},
        description={#6},#1
    }
    \makeglossaries
    \newacronym[see={[Glossaire:]{gls-#3}},#2]{#3}{#4}{#5\glsadd{gls-#3}}
}


\newdualentry{isc}
  {ISC}
  {Institut des Sciences Cognitives Marc Jeannerod}
  {Un centre de recherche dépendant du \gls{CNRS} spécialisé dans la recherche autour du cerveau}

\newdualentry{CNIL}
  {CNIL}
  {Comission Nationale de l'Informatique et des Libertés}
  {Autorité administrative indépendante française chargée de veiller à ce que l’informatique soit au service du citoyen et qu’elle ne porte atteinte ni à l’identité humaine, ni aux droits de l’Homme, ni à la vie privée, ni aux libertés individuelles ou publiques}

\newdualentry{CNRS}
  {CNRS}
  {Centre National de la Recherche Scientifique}
  {Le centre de recherche scientifique civil français}

\newdualentry{SF}
  {SF}
  {science-fiction}
  {Genre narratif consistant à raconter des fictions reposant sur des progrés scientifiques et techniques obtenus dans un futur plus ou moins lointain}

\newdualentry{ecas}
  {ECAS}
  {Entrainement des capacités de Concentration et d'Attention en milieu Scolaire}
  {Le projet de jeu sérieux sur lequel est basé mon stage} 

\newglossaryentry{gabor} {
  name = gabor,
  description = {Stimuli visuel de bas niveau, réalisé en faisant varier la luminance par rapport au fond},
  plural = gabors
}

\newglossaryentry{Psychotoolbox}{
  name = Psychotoolbox,
  description = {Bibliothèque de fonctions pour \gls{Matlab} dédiée à destination de la recherche en neurocience et en vision}
}

\newglossaryentry{Matlab} {
  name = Matlab,
  description = {Langage de programmation destiné aux calculs numériques et utilisé dans la recherche}
}

\newglossaryentry{Kiupe} {
  name = Kiupe,
  description = {Studio de développement de jeux sérieux lyonnais partenaire du projet \gls{ecas}}
}

\begin{document}


\begin{titlepage}
	
	\centering

	\begin{minipage}[b]{0.45\linewidth}
		\centering
		\includegraphics[]{gamagora_logo.png} \\
		5 Avenue Pierre Mendès France, \\
		69676 BRON Cedex \\
		Tél : 04 78 77 30 72 \\
		e.tachet@univ-lyon2.fr
	\end{minipage}
	\hfill
	\begin{minipage}[b]{0.45\linewidth}
		\centering
		\includegraphics[]{isc_logo.jpg} \\
		Institut des Sciences Cognitives Marc Jeannerod \\
		67 bd Pinel, \\
		69675 Bron CEDEX \\
		web@isc.cnrs.fr 
	\end{minipage}
	
	\vspace{2cm}
	
	{\scshape\LARGE Mémoire \par}
	\vspace{1cm}
	{\huge\bfseries Jeu sérieux\par}
	\vspace{0.5cm}
	{\scshape\LARGE ou\par}
	\vspace{0.5cm}
	{\huge\bfseries Création d'un jeu adaptant un protocole expérimental d'entrainement de l'attention\par}
	\vspace{1.5cm}
    {\Large\itshape\textbf{Gwendoline Gomez}\par}
    \vspace{1.5cm}
    {\textbf{Tuteur pédagogique : }M.~\bsc{Didier Puzenat}\\}
    {\textbf{Maître de stage : }Mme.~\bsc{Suliann Ben Hamed}}
	\vfill
	
	\centering 
    \begin{minipage}[b]{0.6\linewidth}
        
    \end{minipage}
    \hfill	
	 \begin{minipage}[b]{0.3\linewidth}
		{\large \today\par}
	 \end{minipage}
\end{titlepage}	

\newpage

\newpage

\section*{Remerciements}

\paragraph{}Je tiens tout d'abord à remercier l'équipe pédagogique de Gamagora, ainsi que les intervenants, qui m’ont permis d’apprendre dans les meilleures conditions durant cette
année et d’ainsi appliquer ces connaissances dans le milieu professionnel durant ce stage.

\paragraph{}Je souhaite remercier tout particulièrement ma maître de stage \bsc{Suliann Ben Hamed} ainsi que toute l'équipe projet, à savoir \bsc{Simon Clavagnier}, \bsc{Camilla Ziane}
et \bsc{Pascal Chabaud}, qui m'ont fait découvrir le monde de la neuroscience et m'ont fait confiance sur ce projet.

\paragraph{}Je remercie également \bsc{Sylvain Maurin} responsable du réseau informatique de l'\gls{isc}, pour ses précieux conseils, qu'ils soient portés sur la technique ou sur ma vie
professionnelle future. Je le remercie aussi pour m'avoir appris les bases du baby foot.

\paragraph{}Je remercie les thésards et stagiaires de l'\gls{isc} que j'ai pu rencontrer pour les discussions très intéressantes et les parties haletantes de baby foot.

\paragraph{}Je remercie enfin mon ami et binôme de stage \bsc{Mathieu Nivoliez} pour ces 4 mois passés à travailler ensemble dans la bonne humeur. Nous avons découvert l'univers de
la neuroscience mais surtout, nous avons beaucoup appris l'un de l'autre. Ce fut une expérience passionnante.

\newpage

\tableofcontents
\newpage

\listoffigures
\newpage

\section*{Introduction}
\addcontentsline{toc}{section}{Introduction}

\newpage

\section{Présentation du contexte}

\subsection{L'Institut des Sciences Cognitives}

Mon stage se déroule à l'\gls{isc} à Bron. Cet institut fait partie du \gls{CNRS}. L'institut regroupe deux départements : le département de neurosciences cognitives et le département
du langage, cerveau et cognition. Chaque département regroupe des équipes de chercheurs qui étudient le comportement du cerveau, notamment son aspect cognitif. Leurs travaux sont
communiqués dans des articles scientifiques détaillant leurs découvertes ou des nouveaux protocoles par exemple. Vous pouvez obtenir plus d'informations sur l'\gls{isc} sur leur site
internet\cite{ISC00}. Je fais partie du département neurosciences cognitives, dont la directrice est le Dr.\emph{Angela Sirigu}. Ce département comporte plusieurs équipes.

\subsection{L'équipe}Je fais partie de l'équipe dirigée par le Dr.\emph{Suliann Ben Hamed}. Cette équipe travaille entre autres sur le projet d'\gls{ecas}. Ce projet vise à trouver
des solutions d'entrainement de l'attention chez des étudiants n'ayant pas de troubles de l'attention avérés.

\paragraph{}Cette équipe est composée de 6 personnes :
\begin{description}
    \item[Suliann Ben Hamed :] C'est la chef d'équipe. C'est elle qui a lancé ce projet de recherche et qui recherche les financements.
    \item[Pascal Chabaud :] Il est associé sur le projet et s'occupe du recrutement.
    \item[Simon Clavagnier :] C'est un post doctorant chargé de poser les outils et les protocoles à utiliser et de développer l'idée.
    \item[Camilla Ziane :] Elle réalise une thèse dont ce projet est le sujet. Elle doit apprendre et transmettre son expérience. Elle analyse les résultats et développe le projet.
    \item[Mathieu Nivoliez et moi-même :] Nous sommes des stagiaires recrutés pour la réalisation d'un jeu sérieux découlant du travail de l'équipe de chercheurs.\\ \\
\end{description}

\paragraph{}Durant ce projet, d'autres personnes de l'\gls{isc} ont été amenées à nous aider. \emph{Sylvain Maurin}, administrateur réseau de l'\gls{isc}, nous a mis à disposition un
environnement de développement permettant la réalisation du jeu sérieux. Les autres chercheurs et étudiants chercheurs de l'\gls{isc} nous ont également aidés en tant que sujets sur
le protocole expérimental, puis en tant que joueurs pour tester notre jeu.


\section{Projet en cours}

\subsection{L'attention}

\paragraph{}L'équipe dans laquelle je travaille étudie particulièrement l'attention. Des hypothèses ont été faites par des chercheurs sur la manière dont elle se manifeste et comment nous
l'utilisons. Le cerveau reçoit en permanence une multitude d'informations de la part de son environnement que l'on appelle des stimulus. Il n'est pas capable de traiter tous ces
stimulus dans la durée et doit donc se focaliser sur les informations les plus importantes. A l'heure du numérique, l'être humain a tendance à perdre sa capacité d'attention dans la
durée. En effet, les smartphones et leur notifications par exemple ont tendance à sortir leur propriétaire assez régulièrement de leur tache en cours. Ce qui entrainerait une chute de
performance sur des taches qui nécessitent une attention prolongée.

\begin{wrapfigure}[10]{r}{6cm}
\includegraphics[width=6cm]{selectiveAttention.png}
\end{wrapfigure}
\paragraph{}D'après nos observations, nous pensons que l'attention serait sélective selon un spectre. Elle peut être spatiale, visuelle, auditive, tactile ... Lorsqu'elle est visuelle
on sait aussi que l'attention n'est pas la même selon si le stimulus est au milieu du champs de vision ou s'il est plutôt à la périphérie.




\subsection{Les taches d'entrainement}

%L'équipe dans laquelle je travaille fait des recherches sur l'entrainement de l'attention chez les sujets sains. En effet, on sait qu'il est possible d'entrainer l'attention des
%personnes atteintes de troubles de l'attention mais des recherches sont en cours pour le vérifier chez des sujets sains.

%Le but de l'entrainement existant est d'améliorer l'attention soutenue.

\newpage

\section{Problématique}

\paragraph{}D'après la recherche, il est possible d'améliorer son attention avec de l'entrainement. L'équipe de \emph{Suliann Ben Hamed} l'a prouvé à l'aide de protocoles expérimentaux
dont les résultats ont été très satisfaisants. Le but est maintenant de pouvoir tester ces protocoles en grandeur nature sur des élèves et d'observer leurs impacts sur leur
concentration en cours, voir même sur leurs résultats. Mais ces protocoles ont néanmoins un gros défaut. Ils ne sont pas attractifs et sont même rébarbatifs. En effet, il est difficile
de travailler sur les notions de d'adaptativité et de flexibilité et surtout d'obtenir un entrainement multidimensionnel avec des tâches réalisées sur \gls{Matlab}. Il n'est donc pas
possible de motiver un élève a consacrer 1h par jour de son temps sur un entrainement comme celui-la. C'est pourquoi l'objectif de mon stage est de trouver un moyen de rendre ces
entrainements attractifs, voir même addictifs. L'idée est de réaliser un jeu vidéo qui ressemble plus ou moins aux protocoles mais qui aurait les mêmes effets. La question qui en
découle est donc :
\begin{quote}
\textbf{Peut-on entrainer l'attention à l'aide d'un jeu vidéo ?}
\end{quote}

A cette question s'ajoutent quelques difficultés à ne pas négliger, la première étant la communication avec les chercheurs. En effet, il est nécessaire de comprendre leurs
recherches afin de bien saisir leurs besoins et de parler "la même langue".

De plus, il est important de prendre en compte les contraintes techniques notamment en matière de réseau et de stockage de données permettant l'analyse des entrainements.


\newpage

\section{Réalisation}

\subsection{Etude du besoin}

\subsection{Mise en place de l'environnement}

\subsection{Prototypage du premier jeu}

\subsection{Gestion des données}

\newpage 

\section*{Conclusion}
\addcontentsline{toc}{section}{Conclusion}

\paragraph{}Le but de ce stage était la réalisation d'un jeu sérieux permettant d'entrainer l'attention en milieu scolaire de n'importe quel étudiant. Nous nous sommes inspirés des
tâches d'entrainement existantes afin de créer un prototype fonctionnel. Celui-ci permet grâce à sa modularité de jouer sur différents aspects de l'attention. Les données de chaque
joueur sont récupérées pour être analysées. Le prochain objectif est de diffuser ce prototype et d'observer son impact sur l'évolution des capacités attentionnelles des joueurs.

\paragraph{}Ce stage a été formateur pour moi pour plusieurs raisons. Il y a tout d'abord le contexte du projet. Je suis très intéressée par les jeux sérieux et ce projet m'a permis de
comprendre les enjeux d'un jeu vidéo à la fois en milieu scolaire, mais aussi dans le cadre de la recherche. Ensuite, le projet en étant à ces débuts, nous avons participé au processus
de création du jeu dans sa globalité, de son concept jusqu'à l'obtention d'un prototype fonctionnel. Nous nous sommes heurtés à différentes problématiques, techniques et fonctionnelles,
que ce soit sur le fonctionnement du jeu ou la gestion des données.

\paragraph{}Ce projet m'ayant beaucoup plus, et ayant eu une proposition de \emph{Suliann Ben Hamed}, j'ai fait le choix de monter un dossier de thèse CIFRE pour pouvoir continuer de
travailler sur ce projet. Cette thèse se fera en collaboration avec l'entreprise Kiupe, partenaire du projet.

\newpage

\printglossary[type=\acronymtype]
\newpage

\printglossary
\newpage

\printbibliography[heading=bibintoc, title={Bibliographie}]

\section*{Annexes}
\addcontentsline{toc}{section}{Annexes}

\newpage

\begin{abstract}

\paragraph{}Un constat des professeurs a été fait concernant la difficulté de leurs élèves à se concentrer sur leurs cours. Une solution possible est de proposer aux étudiants un
entrainement leur permettant d'améliorer leurs capacités d'attention pour pouvoir être plus attentifs en cours. Des outils existent déjà pour les enfants ayant des problèmes
attentionnels avérés, par exemple les hyperactifs, mais ces outils ne sont pas adaptés au reste de la population. Il était donc nécessaire de créer un nouvel outil. Pour être ludique
et attractif auprès de ce public, le choix a été fait de créer un jeu vidéo sérieux. Ce mémoire explique le processus de création de ce jeu réalisé dans le cadre de mon stage de fin
de Master 2 à l'\gls{isc}, depuis les premiers concepts de gameplay jusqu'à la réalisation d'un jeu fonctionnel.

\end{abstract}





\end{document}