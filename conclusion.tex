\section*{Conclusion}
\addcontentsline{toc}{section}{Conclusion}

\paragraph{}Le but de ce stage était la réalisation d'un jeu sérieux permettant d'entrainer l'attention en milieu scolaire de n'importe quel étudiant. Nous nous sommes inspirés des
tâches d'entrainement existantes afin de créer un prototype fonctionnel. Celui-ci permet grâce à sa modularité de jouer sur différents aspects de l'attention. Les données de chaque
joueur sont récupérées pour être analysées. Le prochain objectif est de diffuser ce prototype et d'observer son impact sur l'évolution des capacités attentionnelles des joueurs.

\paragraph{}Ce stage a été formateur pour moi pour plusieurs raisons. Il y a tout d'abord le contexte du projet. Je suis très intéressée par les jeux sérieux et ce projet m'a permis de
comprendre les enjeux d'un jeu vidéo à la fois en milieu scolaire, mais aussi dans le cadre de la recherche. Ensuite, le projet en étant à ces débuts, nous avons participé au processus
de création du jeu dans sa globalité, de son concept jusqu'à l'obtention d'un prototype fonctionnel. Nous nous sommes heurtés à différentes problématiques, techniques et fonctionnelles,
que ce soit sur le fonctionnement du jeu ou la gestion des données.

\paragraph{}Ce projet m'ayant beaucoup plus, et ayant eu une proposition de \emph{Suliann Ben Hamed}, j'ai fait le choix de monter un dossier de thèse CIFRE pour pouvoir continuer de
travailler sur ce projet. Cette thèse se fera en collaboration avec l'entreprise Kiupe, partenaire du projet.

\newpage