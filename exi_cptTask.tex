\subsubsection{La tâche de CPT (Continuous Performance Task)}

\begin{wrapfigure}[7]{l}{2.3cm}
\vspace{-15pt}
\includegraphics[width=2.3cm]{gabor.jpg}
\captionsetup{labelformat=simpleNumber}
\caption{Gabor}
\end{wrapfigure}

\paragraph{Le dispositif}Le joueur est placé à une distance de 57 cm d'un écran grâce à une mentonnière qui permet de garder une certaine stabilité. Il dispose d'un clavier avec
lequel il se sert uniquement des touches \emph{espace} et \emph{1} et \emph{2} du pavé numérique. Des stimuli sous forme de \emph{\glspl{gabor}} sont présentés au sujet. La cible est un gabor
orienté à 45\degre. Les distracteurs sont des gabors dont l'orientation varie entre 6 et 42\degre par rapport à la cible. Les stimuli apparaissent aléatoirement pendant 80 ms. La
cible a une probabilité d'apparition de 12.5\%, soit $1/8$. Pour ne pas donner de rythme à l'apparition des stimuli, un temps d'attente de 200ms à 1 seconde les espace.

\begin{figure}[H]
    \begin{center}
    \includegraphics[width=13cm]{cptDispositif.png}
    \end{center}
    \caption{Dispositif de la tâche de CPT et modélisation d'un gabor}
\label{CptDispositif}
\end{figure}

\newpage
\paragraph{}Le sujet a pour objectif de taper sur la barre espace lorsqu'il reconnait la cible. Il dispose d'une seconde après l'apparition de la cible pour répondre. Nous ne cherchons
pas à mesurer les réflexes du sujet mais sa capacité de concentration. S'il trouve la cible, le sujet entend un bip. S'il tape sur un distracteur, il entend un cri. Le sujet a quatre
possibilités de réponses :
\begin{itemize}
\item Il appuie sur une cible. C'est une \textbf{\emph{détection correcte}} (\emph{hit}).
\item Il n'appuie pas sur un distracteur. C'est une \textbf{\emph{rejection correcte}}.
\item Il appuie sur un distracteur. C'est une \textbf{\emph{fausse alarme}} (\emph{false alarm}).
\item Il n'appuie pas sur une cible. C'est une \textbf{\emph{omission}} (\emph{miss}).
\end{itemize}
Le sujet réalise plusieurs séries dans lesquelles il doit trouver la cible 80 fois. La durée complète de l'entrainement dépend des capacités du sujet. Elle oscille généralement entre
1 et 2h pour pouvoir observer des variations durant l'entrainement et des améliorations au fur et à mesure des sessions. Les séries sont entrecoupées d'un test de discrimination.

\begin{figure}[H]
    \begin{center}
    \includegraphics[width=13cm]{cptTask.png}
    \end{center}
    \caption{Exemple d'une partie de série de la tâche de CPT}
\label{CptTask}
\end{figure}

\paragraph{Le test de discrimination}Tout le monde n'a pas la même sensibilité visuelle. Il est donc nécessaire de déterminer le seuil de perception visuelle du sujet afin de calibrer
ses résultats en fonction. Le dispositif est le même que pour la tâche de CPT. La cible et les distracteurs sont les mêmes. On demande au sujet de fixer un point central et on affiche
des couples cible/distracteur présentés aléatoirement. Un premier stimulus est affiché pendant 80ms, suivi d'un masque pendant 50ms. Ce masque symbolise le gabor suivant dans la
tâche de CPT. Il augmente la difficulté car il empêche toute analyse du gabor postérieure à son apparition. Puis un écran vierge est présenté pendant 1 secondes. On affiche ensuite
le deuxième stimulus du couple de la même manière que le premier. Le sujet doit alors dire si la cible a été présentée en premier ou en deuxième avec les touches \emph{1} et \emph{2}
du pavé numérique. Il a 1.5 secondes pour répondre. Comme pour la tâche de CPT, s'il a juste, il entend un bip, sinon un cri. Les couples cibles/distracteurs peuvent apparaitre soit
au milieu de l'écran en vision centrale, soit légèrement décalés à droite ou à gauche, à la vision périphérique du sujet. Le test de discrimination dure environ 6 minutes, puis une
nouvelle série de la tâche de CPT est lancée.

\begin{figure}[H]
    \begin{center}
    \includegraphics[width=13cm]{cptDiscriminationTask.png}
    \end{center}
    \caption{Exemple d'une partie de test de discrimination}
\label{CptDiscriminationTask}
\end{figure}


\paragraph{Traitement des données}Les données collectées sont propres à chaque sujet et sont analysées par session. On peut ainsi voir l'impact de l'entrainement sur le
sujet et comparer son évolution par rapport aux autres. J'ai pu participer à la tâche d'entrainement en tant que sujet afin de bien comprendre son fonctionnement. \`{A}
chaque session, mes performances se sont améliorées. La figure \ref{CptPerformance} représente mes résultats sur 5 sessions.

\begin{figure}[H]
    \begin{center}
    \includegraphics[width=11cm]{cptPerformance.png}
    \end{center}
    \caption{L'évolution de mes performances sur la tâche de CPT}
\label{CptPerformance}
\end{figure}

\paragraph{}Si l'on regarde le pourcentage de hits en rouge et celui du score en noir, on remarque qu'ils augmentent significativement au fur et à mesure des sessions. Le nombre de FA
(bleu) au contraire est en baisse alors que le nombre de fois où la touche espace est pressée (rose) ne bouge pas beaucoup. On constate aussi que la distribution des erreurs, qui
correspond aux résultats du test de discrimination, ne bouge pas beaucoup. Cela veut dire que ce n'est pas ma capacité à reconnaitre la cible qui a changée mais plutôt mes capacités
de concentration et de contrôle sur mon attention.