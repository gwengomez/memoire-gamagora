\section*{Introduction}
\addcontentsline{toc}{section}{Introduction}

\paragraph{}Dans la société actuelle, les étudiants ont accès à tout et n'importe quoi, à n'importe quel moment. Ils sont entourés d'un nombre très important de stimuli, comme les
smartphones et leurs notifications par exemple. Leur cerveau reçoit un trop grand nombre de stimuli et n'est pas capable de donner lui-même plus ou moins d'importance à une tâche.
C'est pourquoi les professeurs ont fait un constat : tout est prétexte pour que les élèves se déconcentrent de leurs cours. Ils n'arrivent pas à rester concentrés en profondeur sur
une tâche mais sautent d'une tâche à l'autre, ce qui implique une baisse des capacités de temps de concentration. On observe une évolution des capacité de temps de concentration au
fur et à mesure de l'âge : un enfant de 2 ans ne peut pas se concentrer plus de 5 minutes tandis qu'un adulte devrait pouvoir se concentrer pendant 45 minutes environ.

\paragraph{}Cette complainte des professeurs a donné lieu à des recherches pour créer un outil en milieu éducatif qui pourrait stimuler l'attention de n'importe quel jeune. Le but
est de pouvoir rallonger la durée de concentration des étudiants, mais aussi de leur permettre de contrôler l'importance donnée à différents stimuli, qu'ils puissent faire la part des
priorités.

\paragraph{}Beaucoup d'entrainements existent mais ceux-ci sont à destination des enfants hyperactifs. Le but de ces entrainements est de les aider à contrôler leur attention en leur
demandant de s'empêcher de faire une action. Les problèmes attentionnels ne sont pas spécifiques aux hyperactifs. Notre environnement étant très riche, tout le monde est concerné.
Cependant les entrainements pour hyperactifs ne sont pas adapté pour notre population cible car il est nécessaire d'améliorer aussi les temps de concentration et la priorité donnée
aux différents stimuli.

\paragraph{}En réponse à ces problématiques, il nous a été demandé de réaliser cet outil d'entrainement sous la forme d'un jeu sérieux. L'avantage de ce format est qu'il est ludique et
touche plus facilement les jeunes. On peut les attirer et les faire s'entrainer sans qu'ils s'en rendent compte.

\newpage