\section{Présentation}

\subsection{L'Institut des Sciences Cognitives}

Mon stage se déroule à l'\gls{isc} à Bron. Cet institut fait partie du\gls{CNRS}. L'institut regroupe deux départements : le département de neurosciences cognitives et le département
du langage, cerveau et cognition. Chaque département regroupe des équipes de chercheurs qui étudient le comportement du cerveau, notamment de son aspect cognitif. Leurs travaux peuvent
aboutir sur des papiers de recherche pouvant détailler des découvertes ou des nouveaux protocoles par exemple. Vous pouvez obtenir plus d'informations sur l'\gls{isc} sur leur site
internet\cite{ISC00}. Je fais partie du département neurosciences cognitives, dont la directrice est \emph{Angela Sirigu}. Ce département comporte plusieurs équipes dont la plupart
font des recherches sur les primates.

\subsection{L'équipe}Je fais partie de l'équipe dirigée par \emph{Suliann Ben Hamed} dans le département d'Angela. Cette équipe travaille sur le projet d'\gls{ecas}. Ce projet vise à trouver
des solutions d'entrainement de l'attention chez des étudiants n'ayant pas de troubles de l'attention avérés.

\paragraph{}Cette équipe est composée de 6 personnes :
\begin{description}
    \item[Suliann Ben Hamed :] C'est la chef d'équipe. C'est elle qui a lancé ce projet de recherche.
    \item[Pascal Chabaud :] Il est associé sur le projet. Il s'occupe du recrutement sur le projet et recherche les financements.
    \item[Simon Clavagnier :] C'est un post doctorant chargé de poser les outils et les protocoles à utilise et de développer l'idée.
    \item[Camilla Ziane :] Elle réalise une thèse dont ce projet est le sujet. Elle doit apprendre et transmettre son expérience sur ce projet. Elle analyse les résultats et développe
    le projet.
    \item[Mathieu Nivoliez et moi-même :] Nous sommes des stagiaires recrutés pour la réalisation d'un jeu sérieux découlant du travail de l'équipe de chercheurs.\\ \\
\end{description}

\paragraph{}Durant ce projet, d'autres personnes de l'\gls{isc} ont été amenées à nous aider. \emph{Sylvain Maurin}, administrateur réseau de l'\gls{isc}, nous a mis à disposition un
environnement de développement permettant la réalisation du jeu sérieux. Les autres chercheurs de l'\gls{isc} nous ont également aidé en en tant que sujets sur le protocole
expérimental, puis en tant que joueurs pour tester notre jeu.
