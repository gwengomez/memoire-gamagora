\section{Présentation du contexte}

\subsection{L'Institut des Sciences Cognitives}

Mon stage se déroule à l'\gls{isc} à Bron. Cet institut fait partie du \gls{CNRS}. L'institut regroupe deux départements : le département de neurosciences cognitives et le département
du langage, cerveau et cognition. Chaque département regroupe des équipes de chercheurs qui étudient le comportement du cerveau, notamment son aspect cognitif. Leurs travaux sont
communiqués dans des articles scientifiques détaillant leurs découvertes ou des nouveaux protocoles par exemple. Vous pouvez obtenir plus d'informations sur l'\gls{isc} sur leur site
internet\cite{ISC00}. Je fais partie du département neurosciences cognitives, dont la directrice est le Dr.\emph{Angela Sirigu}. Ce département comporte plusieurs équipes.

\subsection{L'équipe}Je fais partie de l'équipe dirigée par le Dr.\emph{Suliann Ben Hamed}. Cette équipe travaille entre autres sur le projet d'\gls{ecas}. Ce projet vise à trouver
des solutions d'entrainement de l'attention chez des étudiants n'ayant pas de troubles de l'attention avérés.

\paragraph{}Cette équipe est composée de 6 personnes :
\begin{description}
    \item[Suliann Ben Hamed :] C'est la chef d'équipe. C'est elle qui a lancé ce projet de recherche et qui recherche les financements.
    \item[Pascal Chabaud :] Il est associé sur le projet et s'occupe du recrutement.
    \item[Simon Clavagnier :] C'est un post doctorant chargé de poser les outils et les protocoles à utiliser et de développer l'idée.
    \item[Camilla Ziane :] Elle réalise une thèse dont ce projet est le sujet. Elle doit apprendre et transmettre son expérience. Elle analyse les résultats et développe le projet.
    \item[Mathieu Nivoliez et moi-même :] Nous sommes des stagiaires recrutés pour la réalisation d'un jeu sérieux découlant du travail de l'équipe de chercheurs.\\ \\
\end{description}

\paragraph{}Durant ce projet, d'autres personnes de l'\gls{isc} ont été amenées à nous aider. \emph{Sylvain Maurin}, administrateur réseau de l'\gls{isc}, nous a mis à disposition un
environnement de développement permettant la réalisation du jeu sérieux. Les autres chercheurs et étudiants chercheurs de l'\gls{isc} nous ont également aidés en tant que sujets sur
le protocole expérimental, puis en tant que joueurs pour tester notre jeu.
