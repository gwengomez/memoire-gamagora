\section{Problématique}

\paragraph{}D'après la recherche, il est possible d'améliorer son attention avec de l'entrainement. L'équipe de \emph{Suliann Ben Hamed} l'a prouvé à l'aide de protocoles expérimentaux
dont les résultats ont été très satisfaisants. Le but est maintenant de pouvoir tester ces protocoles en grandeur nature sur des élèves et d'observer leurs impacts sur leur
concentration en cours, voir même sur leurs résultats. Mais ces protocoles ont néanmoins un gros défaut. Ils ne sont pas attractifs et sont même rébarbatifs. En effet, il est difficile
de travailler sur les notions de d'adaptativité et de flexibilité et surtout d'obtenir un entrainement multidimensionnel avec des tâches réalisées sur \gls{Matlab}. Il n'est donc pas
possible de motiver un élève a consacrer 1h par jour de son temps sur un entrainement comme celui-la. C'est pourquoi l'objectif de mon stage est de trouver un moyen de rendre ces
entrainements attractifs, voir même addictifs. L'idée est de réaliser un jeu vidéo qui ressemble plus ou moins aux protocoles mais qui aurait les mêmes effets. La question qui en
découle est donc :
\begin{quote}
\textbf{Peut-on entrainer l'attention à l'aide d'un jeu vidéo ?}
\end{quote}

A cette question s'ajoutent quelques difficultés à ne pas négliger, la première étant la communication avec les chercheurs. En effet, il est nécessaire de comprendre leurs
recherches afin de bien saisir leurs besoins et de parler "la même langue".

De plus, il est important de prendre en compte les contraintes techniques notamment en matière de réseau et de stockage de données permettant l'analyse des entrainements.


\newpage