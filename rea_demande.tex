\subsection{Demande du client}

\paragraph{}Dans le cadre de la réalisation du projet, des discussions préliminaires sur les attentes des chercheurs ont d'abord eu lieu. Ils avaient quelques idées, donc nous avons
pu partir sur cette base pour pouvoir proposer des solutions. Voici les demandes qui nous ont été faites :


\paragraph{1}Pour que les joueurs jouent régulièrement à notre jeu, il faut que celui soit assez attractif.
\paragraph{2}L'attention comporte plusieurs aspects (spatiale, divisive, exécutive ...). Il faudrait dans le jeu un monde pour chaque aspect. Le joueur commencerai par un mode simple,
exigeant seulement son attention soutenue, puis les monde suivant intégreraient petit à petit les autres aspects. Il est néanmoins nécessaire de choisir l'ordre des mondes, dont la
difficulté n'est pas forcément la même pour tout le monde.
\paragraph{3}Généraliser le contexte : si un joueur améliore ses capacités seulement dans un contexte précis, rien ne nous dit qu'il va s'en servir dans d'autres contextes, que ce soit
dans d'autres jeux ou dans la vie de tous les jours.
\paragraph{4}La difficulté doit être adaptative. En effet, chacun à ses propres capacités attentionnelles. Pour que les joueurs puissent progresser, il est nécessaire que le jeu
s'adapte à leur rythme d'amélioration de l'attention pour leur proposer la bonne difficulté de jeu.
\paragraph{5}Il est nécessaire de recueillir les données des joueurs pour pouvoir analyser leur progression attentionnelle.
\paragraph{6}Ce projet est réalisé dans le cadre de l'éducation nationale. La cible principale du jeu sera donc les élèves au collège et lycée. L'idée est de proposer aux professeurs de
faire jouer leurs élève en cours, 1h par semaine. Il faut donc trouver une solution pour que le projet les intéresse.
\paragraph{7}Les dynamiques de groupe aident souvent à progresser dans beaucoup de domaines. Il serait donc intéressant de proposer un mode multijoueur dans lequel un groupe de 3 ou 4
étudiants joueraient en coopération pour atteindre un but. La compétition aidant aussi, on peut imaginer que les groupes joueraient les uns contre les autres. Dans ce contexte, il est
important de ne pas oublier d'inclure le professeur dans la boucle de jeu pour qu'il puisse interagir avec ses élèves.
\paragraph{8}L'\gls{isc} doit pouvoir garder la main sur le code source en cas de partenariat avec un studio de jeu.