\subsection{Prototypage du premier jeu}


\subsubsection{Le concept}


\begin{wrapfigure}[15]{l}{5cm}
    \vspace{-15pt}
    \includegraphics[width=5cm]{temps-reaction.jpg}
    \captionsetup{labelformat=simpleNumber}
    \caption{Temps de réaction}
\end{wrapfigure}

\paragraph{}Nous avons choisi le mini jeu du tuyau comme premier jeu car l'idée était plutôt simple. Il y a un tuyau au milieu de l'écran qui est brisé en son centre. Le
joueur voit des objets tomber par le trou causé par la brisure. Il doit identifier un objet cible parmi tous les autres pour l'enlever du tuyau. Pour éviter de jouer sur ses réflexes,
il faut qu'il puisse interagir avec les objets un peu de temps après qu'ils ne soient plus visibles. Pour cela, le joueur doit pouvoir interagir avec les objets qu'à partir du moment
où il peut les voir, et jusqu'à ce qu'ils soient détruits en atteignant le bas de l'écran.\\ \\

\paragraph{}A chaque fois que le joueur enlève un objet cible, ses points de score augmentent, s'il en loupe ou s'il essaie d'enlever le mauvais objet, ses points de score baissent.
Nous n'étions pas encore sûrs des conditions de victoire :
\begin{itemize}
\item Est-ce que le joueur doit atteindre un certain score avant la fin du temps ?
\item Est-ce que le joueur doit faire le meilleur score en un laps de temps imparti ?
\item Est-ce que le joueur a droit à un nombre d'erreurs maximum ?
\end{itemize}


\paragraph{}Il était également important de définir quels étaient les paramètres à régler. Nous avons pu dégager une première liste de paramètres au préalable, certains
quantifiables, d'autres non.

Parmi les quantifiables :
\begin{itemize}
    \item Vitesse d'apparition/disparition des éléments
    \item Nombre de cibles total ou pourcentage de chance d'apparition de la cible durant la partie
    \item Nombre de distracteurs total ou pourcentage de chance d'apparition du distracteur durant la partie
    \item Emplacement des éléments par rapport au centre
    \item Nombre de cible a trouver pour finir la partie
    \item Temps alloué à la partie
    \item Nombre de distracteurs différents\\
\end{itemize}
Parmi les inquantifiables :
\begin{itemize}
    \item Le design de la cible
    \item Le design des distracteurs
    \item La ressemblance entre la cible et les distracteurs
\end{itemize}

\paragraph{}Pour les touches nous avons choisi pour commencer de garder les mêmes touches que pour la tâche de CPT. Pour enlever les objets, il faudrait alors utiliser la barre espace.
S'il y a plusieurs objets à enlever, cela choisira le premier apparut, donc le plus bas.

\subsubsection{Création de la première scène}

\paragraph{}Pour notre prototype, nous avons commencé par réaliser une première scène plutôt simple. Quelques cylindres représentent le tuyau. Le design n'étant pas très important
dans un premier temps, nous avons fait quelque chose de simple ressemblant un peu à Mario. Un bouton "Start" lance le jeu, remplacé par un bouton "Stop" si l'on veut l'arrêter. Sinon,
la partie continue jusqu'à ce qu'un certain nombre de cibles soit apparues. \\

\begin{figure}[H]
    \begin{center}
    \includegraphics[width=10cm]{proto-pipe1.png}
    \end{center}
    \caption{1\up{er} prototype}
\label{ProtoPipe1}
\end{figure}

\begin{wrapfigure}[9]{l}{3cm}
    \vspace{-10pt}
    \includegraphics[width=3cm]{proto-pipe-elements.png}
    \captionsetup{labelformat=simpleNumber}
    \caption{Objets}
\end{wrapfigure}

\paragraph{}Les objets dont le joueur doit faire attention sont des boules bleues, plus ou moins claires. La boule la plus claire est la cible, les autres sont les distracteurs. Le
joueur doit être attentif pour enlever la bonne boule. Lorsqu'il en enlève une, celle-ci se déplace jusqu'au point de score pour montrer qu'il a réussi, et le score s'incrémente.
S'il réussit à enlever une cible, il gagne 15 points. S'il en rate une ou s'il essaie d'en enlever une alors qu'il n'y en a pas, il perd 5 points.

\newpage

\paragraph{}Lors d'une partie, nous mesurons la performance d'une personne. Nous en parlerons plus en détail dans la partie \ref{Donnees}. Selon la performance du joueur, nous pouvons
augmenter ou diminuer la difficulté de plusieurs manières.

\paragraph{}D'abord, nous avons cherché à refermer le tube pour que les objets soient visibles pendant moins longtemps. La partie basse du tube peut monter jusqu'à un certain point, les objets
devant rester visibles. Le joueur doit donc être plus attentif.

\paragraph{}Puis, comme le jeu était toujours assez facile, nous avons fait bouger les tubes. Cela n'est pas très clair sur la figure \ref{ProtoPipeDifficultes} mais lorsque le joueur est
performant, le tube se met à trembler tout en pivotant de quelques crans sur la droite ou la gauche. Nous avons mis un angle limite pour éviter une rotation complète du tube.

\paragraph{}Enfin, le jeu étant encore trop facile, nous avons rajouté la possibilité d'ajouter des tuyaux. Il est possible d'en rajouter jusqu'à 5 (ce qu'il est possible d'afficher à l'écran). Le
nombre de tuyaux reste fixe durant une partie, le joueur ne pouvant en rajouter ou en enlever qu'entre deux parties.

\paragraph{}Durant une partie, tous ces paramètres de difficultés peuvent être actifs en même temps. Néanmoins, ceux-ci s'activent selon la performance du joueur : si celui-ci n'est pas très
performant, la difficulté ne bougera pas trop, voire baissera pour s'adapter à son niveau. \\ \\

\begin{figure}[H]
    \begin{center}
    \includegraphics[width=13cm]{proto-pipe-difficultes-evolution.png}
    \end{center}
    \caption{Evolution de la difficulté}
\label{ProtoPipeDifficultes}
\end{figure}

\newpage
\subsubsection{Amélioration du design et du gameplay}

\paragraph{}Après que le prototype soit fonctionnel et validé par les chercheurs, nous avons voulu le rendre un peu plus esthétique. Nous en avons aussi profité pour améliorer le
gameplay et rajouter du contenu. \\


\begin{wrapfigure}[13]{l}{6cm}
    \vspace{-10pt}
    \includegraphics[width=6cm]{hl2-power-generator.jpg}
    \captionsetup{labelformat=simpleNumber}
    \caption{Décor}
\end{wrapfigure}

\paragraph{Contexte}Notre thème est la \gls{SF}. Nous avons donc cherché des inspirations dans des jeux avec cette thématique. L'un d'eux nous est venu assez rapidement à l'esprit.
Il y a dans Half Life 2 des générateurs d'énergie qui ressemble à des tubes dans laquelle une sphère d'énergie flotte. Ces générateurs sont d'abord fermés par une sorte de bouclier
qu'il faut ouvrir pour pouvoir récupérer la sphère d'énergie. Nous avons voulu nous baser sur ce principe pour notre gameplay.

\paragraph{}Nos tubes de Mario sont donc devenus des tubes électromagnétiques, protégés par des boucliers qui peuvent s'ouvrirent en se séparant en deux au milieu. Des sphères
d'énergie descendant le long du tube remplacent les boules bleues et sont maintenant les nouveaux distracteurs et cible.

\begin{figure}[H]
    \begin{center}
    \includegraphics[width=9cm]{PDC-tube-evolution.jpg}
    \end{center}
    \caption{Création des tubes}
\label{TubeEvolution}
\end{figure}

\begin{wrapfigure}[5]{r}{5cm}
    \vspace{-25pt}
    \includegraphics[width=5cm]{PDC-decor.png}
    \captionsetup{labelformat=simpleNumber}
    \caption{Décor}
\end{wrapfigure}

\paragraph{}En parallèle, nous avons construit une scène Unity avec un package de \gls{SF} pour ambiancer le mini jeu. Cette scène représente l'intérieur d'une base scientifique avec
des couleurs plutôt blanches, bleues et oranges.

\newpage

\paragraph{Gameplay}Pour pouvoir intégrer le tube dans la scène, nous avons pensé à le multiplier et le placer sur un barillier qui pourrait tourner. Chaque tube aurait un niveau
d'énergie. Le joueur ne pourrait interagir qu'avec certains tubes à la fois. Il ne faut pas en mettre trop car sinon, cela deviendrait bien trop difficile. Le but du joueur est
d'enlever des sphère d'énergie instable (les cibles) pour ne pas que le niveau d'énergie stable d'un des tubes arrive à 0.

\begin{figure}[H]
    \begin{center}
    \includegraphics[width=13cm]{PDC-barillier-evolution.jpg}
    \end{center}
    \caption{Barillier de tubes}
\label{BarillierTube}
\end{figure}


\label{prototypage}